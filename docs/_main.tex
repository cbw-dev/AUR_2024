% Options for packages loaded elsewhere
\PassOptionsToPackage{unicode}{hyperref}
\PassOptionsToPackage{hyphens}{url}
%
\documentclass[
]{book}
\usepackage{amsmath,amssymb}
\usepackage{iftex}
\ifPDFTeX
  \usepackage[T1]{fontenc}
  \usepackage[utf8]{inputenc}
  \usepackage{textcomp} % provide euro and other symbols
\else % if luatex or xetex
  \usepackage{unicode-math} % this also loads fontspec
  \defaultfontfeatures{Scale=MatchLowercase}
  \defaultfontfeatures[\rmfamily]{Ligatures=TeX,Scale=1}
\fi
\usepackage{lmodern}
\ifPDFTeX\else
  % xetex/luatex font selection
\fi
% Use upquote if available, for straight quotes in verbatim environments
\IfFileExists{upquote.sty}{\usepackage{upquote}}{}
\IfFileExists{microtype.sty}{% use microtype if available
  \usepackage[]{microtype}
  \UseMicrotypeSet[protrusion]{basicmath} % disable protrusion for tt fonts
}{}
\makeatletter
\@ifundefined{KOMAClassName}{% if non-KOMA class
  \IfFileExists{parskip.sty}{%
    \usepackage{parskip}
  }{% else
    \setlength{\parindent}{0pt}
    \setlength{\parskip}{6pt plus 2pt minus 1pt}}
}{% if KOMA class
  \KOMAoptions{parskip=half}}
\makeatother
\usepackage{xcolor}
\usepackage{color}
\usepackage{fancyvrb}
\newcommand{\VerbBar}{|}
\newcommand{\VERB}{\Verb[commandchars=\\\{\}]}
\DefineVerbatimEnvironment{Highlighting}{Verbatim}{commandchars=\\\{\}}
% Add ',fontsize=\small' for more characters per line
\usepackage{framed}
\definecolor{shadecolor}{RGB}{248,248,248}
\newenvironment{Shaded}{\begin{snugshade}}{\end{snugshade}}
\newcommand{\AlertTok}[1]{\textcolor[rgb]{0.94,0.16,0.16}{#1}}
\newcommand{\AnnotationTok}[1]{\textcolor[rgb]{0.56,0.35,0.01}{\textbf{\textit{#1}}}}
\newcommand{\AttributeTok}[1]{\textcolor[rgb]{0.13,0.29,0.53}{#1}}
\newcommand{\BaseNTok}[1]{\textcolor[rgb]{0.00,0.00,0.81}{#1}}
\newcommand{\BuiltInTok}[1]{#1}
\newcommand{\CharTok}[1]{\textcolor[rgb]{0.31,0.60,0.02}{#1}}
\newcommand{\CommentTok}[1]{\textcolor[rgb]{0.56,0.35,0.01}{\textit{#1}}}
\newcommand{\CommentVarTok}[1]{\textcolor[rgb]{0.56,0.35,0.01}{\textbf{\textit{#1}}}}
\newcommand{\ConstantTok}[1]{\textcolor[rgb]{0.56,0.35,0.01}{#1}}
\newcommand{\ControlFlowTok}[1]{\textcolor[rgb]{0.13,0.29,0.53}{\textbf{#1}}}
\newcommand{\DataTypeTok}[1]{\textcolor[rgb]{0.13,0.29,0.53}{#1}}
\newcommand{\DecValTok}[1]{\textcolor[rgb]{0.00,0.00,0.81}{#1}}
\newcommand{\DocumentationTok}[1]{\textcolor[rgb]{0.56,0.35,0.01}{\textbf{\textit{#1}}}}
\newcommand{\ErrorTok}[1]{\textcolor[rgb]{0.64,0.00,0.00}{\textbf{#1}}}
\newcommand{\ExtensionTok}[1]{#1}
\newcommand{\FloatTok}[1]{\textcolor[rgb]{0.00,0.00,0.81}{#1}}
\newcommand{\FunctionTok}[1]{\textcolor[rgb]{0.13,0.29,0.53}{\textbf{#1}}}
\newcommand{\ImportTok}[1]{#1}
\newcommand{\InformationTok}[1]{\textcolor[rgb]{0.56,0.35,0.01}{\textbf{\textit{#1}}}}
\newcommand{\KeywordTok}[1]{\textcolor[rgb]{0.13,0.29,0.53}{\textbf{#1}}}
\newcommand{\NormalTok}[1]{#1}
\newcommand{\OperatorTok}[1]{\textcolor[rgb]{0.81,0.36,0.00}{\textbf{#1}}}
\newcommand{\OtherTok}[1]{\textcolor[rgb]{0.56,0.35,0.01}{#1}}
\newcommand{\PreprocessorTok}[1]{\textcolor[rgb]{0.56,0.35,0.01}{\textit{#1}}}
\newcommand{\RegionMarkerTok}[1]{#1}
\newcommand{\SpecialCharTok}[1]{\textcolor[rgb]{0.81,0.36,0.00}{\textbf{#1}}}
\newcommand{\SpecialStringTok}[1]{\textcolor[rgb]{0.31,0.60,0.02}{#1}}
\newcommand{\StringTok}[1]{\textcolor[rgb]{0.31,0.60,0.02}{#1}}
\newcommand{\VariableTok}[1]{\textcolor[rgb]{0.00,0.00,0.00}{#1}}
\newcommand{\VerbatimStringTok}[1]{\textcolor[rgb]{0.31,0.60,0.02}{#1}}
\newcommand{\WarningTok}[1]{\textcolor[rgb]{0.56,0.35,0.01}{\textbf{\textit{#1}}}}
\usepackage{longtable,booktabs,array}
\usepackage{calc} % for calculating minipage widths
% Correct order of tables after \paragraph or \subparagraph
\usepackage{etoolbox}
\makeatletter
\patchcmd\longtable{\par}{\if@noskipsec\mbox{}\fi\par}{}{}
\makeatother
% Allow footnotes in longtable head/foot
\IfFileExists{footnotehyper.sty}{\usepackage{footnotehyper}}{\usepackage{footnote}}
\makesavenoteenv{longtable}
\usepackage{graphicx}
\makeatletter
\def\maxwidth{\ifdim\Gin@nat@width>\linewidth\linewidth\else\Gin@nat@width\fi}
\def\maxheight{\ifdim\Gin@nat@height>\textheight\textheight\else\Gin@nat@height\fi}
\makeatother
% Scale images if necessary, so that they will not overflow the page
% margins by default, and it is still possible to overwrite the defaults
% using explicit options in \includegraphics[width, height, ...]{}
\setkeys{Gin}{width=\maxwidth,height=\maxheight,keepaspectratio}
% Set default figure placement to htbp
\makeatletter
\def\fps@figure{htbp}
\makeatother
\setlength{\emergencystretch}{3em} % prevent overfull lines
\providecommand{\tightlist}{%
  \setlength{\itemsep}{0pt}\setlength{\parskip}{0pt}}
\setcounter{secnumdepth}{5}
\usepackage{booktabs}

\usepackage{color}
\usepackage{framed}
\setlength{\fboxsep}{.8em}

% These colours were manually entered, they shouldn't matter unless you want pdf output

\newenvironment{redbox}{
  \definecolor{shadecolor}{RGB}{243, 154, 157}
  \color{white}
  \begin{shaded}}
 {\end{shaded}}

\newenvironment{bluebox}{
  \definecolor{shadecolor}{RGB}{172, 210, 237}
  \color{white}
  \begin{shaded}}
 {\end{shaded}}

\newenvironment{greenbox}{
  \definecolor{shadecolor}{RGB}{141, 181, 128}
  \color{white}
  \begin{shaded}}
 {\end{shaded}}
\ifLuaTeX
  \usepackage{selnolig}  % disable illegal ligatures
\fi
\usepackage[]{natbib}
\bibliographystyle{plainnat}
\usepackage{bookmark}
\IfFileExists{xurl.sty}{\usepackage{xurl}}{} % add URL line breaks if available
\urlstyle{same}
\hypersetup{
  pdftitle={NAME OF WORKSHOP YEAR},
  pdfauthor={Faculty: INSTRUCTOR AND TA NAMES},
  hidelinks,
  pdfcreator={LaTeX via pandoc}}

\title{NAME OF WORKSHOP YEAR}
\author{Faculty: INSTRUCTOR AND TA NAMES}
\date{DATES}

\begin{document}
\maketitle

{
\setcounter{tocdepth}{1}
\tableofcontents
}
\part{Introduction}\label{part-introduction}

\chapter{Workshop Info}\label{workshop-info}

Welcome to the YEAR WORKSHOP Canadian Bioinformatics Workshop webpage!

\section{Schedule}\label{schedule}

YOUR SCHEDULE HERE

\section{Pre-work}\label{pre-work}

\href{LINK\%20TO\%20PREWORK}{You can find your pre-work here.}

\chapter{Meet Your Faculty}\label{meet-your-faculty}

\subsubsection{Shraddha Pai}\label{shraddha-pai}

\begin{quote}
Investigator I, OICR
Assistant Professor, University of Toronto
\end{quote}

Dr.~Pai integrates genomics and computational methods to advance precision
medicine. Her previous work involves DNA methylome-based biomarker discovery in psychosis, and building machine learning algorithms for patient classification from multi-modal data. The Pai Lab at the Ontario Institute for Cancer Research focuses on biomarker discovery for detection, diagnosis and prognosis in brain cancers and other brain-related disorders.

\subsubsection{Chaitra Sarathy}\label{chaitra-sarathy}

\begin{quote}
Bioinformatics Specialist
Krembil Research Institute
\end{quote}

Dr.~Sarathy is a computational biologist with industry experience in software development. Her previous research centered around developing multi-scale mathematical models of human systems to characterise biochemical changes in obesity. In addition, she has developed methods based on machine learning and multi-omics integration to identify drug targets in cancer and stratify patients for clinical trials. She currently focusses on characterising genetic malfunctions in neurological diseases.

\subsubsection{Ian Cheong}\label{ian-cheong}

\begin{quote}
MSc. Candidate
University of Toronto
\end{quote}

Ian is a Master's level candidate in the Department of Medical Biophysics at the University of Toronto. His thesis work in the Pai lab involves analysis of single-cell transcriptomes to find the link between brain development and the development of childhood brain cancer.

\subsubsection{Zoe Klein}\label{zoe-klein}

\begin{quote}
PhD. Candidate
University of Toronto
\end{quote}

Zoe is a PhD level candidate in the Department of Molecular Genetics at the University of Toronto. Her thesis work in the Reimand lab involves using computational tools to investigate the role of non-coding RNA in cancer.

\subsubsection{Nia Hughes (she/her)}\label{nia-hughes-sheher}

\begin{quote}
Platform Training Manager, Canadian Bioinformatics Hub
Ontario Institute for Cancer Research
Toronto, ON, Canada

--- \href{mailto:training@bioinformatics.ca}{\nolinkurl{training@bioinformatics.ca}}
\end{quote}

Nia is the Platform Training Manager for the Canadian Bioinformatics Hub, where she coordinates the Canadian Bioinformatics Workshop Series. Prior to starting at OICR, she completed her M.Sc. in Bioinformatics from the University of Guelph in 2020 before working there as a bioinformatician studying epigenetic and transcriptomic patterns across maize varieties.

\chapter{Data and Compute Setup}\label{data-and-compute-setup}

\subsubsection{Course data downloads}\label{course-data-downloads}

Coming soon!

\subsubsection{Compute setup}\label{compute-setup}

Coming soon!

\part{Modules}\label{part-modules}

\chapter{Module 1}\label{module-1}

\section{Lecture}\label{lecture}

Here is an example of a pdf embedded:

\includegraphics[width=1\textwidth,height=9.375in]{content-files/sample-pdf.pdf}~

Here is an example of a YouTube video embedded:

\subsection{Downloads}\label{downloads}

{[}insert your downloads for this module here (ex. datasets){]}

\section{Lab}\label{lab}

{[}Your lab here{]}

\begin{Shaded}
\begin{Highlighting}[]
\CommentTok{\# Your R code here}

\CommentTok{\# For example:}
\NormalTok{x }\OtherTok{\textless{}{-}} \DecValTok{42}
\NormalTok{x}
\end{Highlighting}
\end{Shaded}

\begin{verbatim}
## [1] 42
\end{verbatim}

\begin{Shaded}
\begin{Highlighting}[]
\CommentTok{\# Your python code here}

\CommentTok{\# For example:}
\BuiltInTok{print}\NormalTok{(}\StringTok{"hello world"}\NormalTok{)}
\end{Highlighting}
\end{Shaded}

\begin{verbatim}
## hello world
\end{verbatim}

\begin{Shaded}
\begin{Highlighting}[]
\CommentTok{\# Your bash code here}

\CommentTok{\# For example:}
\CommentTok{\#pwd}
\end{Highlighting}
\end{Shaded}

Try running these code ``chunks'' by pressing the green (left-pointing) triangle next to your code chunks.

You will see the code run in the console and the output provided below the code chunk.

The output of the code will also be produced under the code chunk on your website page.

  \bibliography{book.bib,packages.bib}

\end{document}
